\documentclass[aspectratio=169,14pt,usenames,dvipsnames]{beamer}

\usepackage[utf8]{inputenc}
\usepackage{fontspec}
\usepackage{enumitem}
\usepackage{calc}

\usepackage{datetime}
\newcommand\builddate{%
   \ifcase \month%
        \or Janeiro%
        \or Fevereiro%
        \or Março%
        \or Abril%
        \or Maio%
        \or Junho%
        \or Julho%
        \or Agosto%
        \or Setembro%
        \or Outubro%
        \or Novembro%
        \or Dezembro%
    \fi\space\number\year%
}

\newcommand{\loadtheme}[1]{%
    \input{themes/#1}%
}
\newcommand{\presentationlanguage}[1]{%
    \usepackage[#1]{babel}%
}

\newcommand{\usecodingsamples}[1]{%
    \usepackage{listings}%
    \input{listings/#1}%
}

% Configura a apresentação para ser executada em tela cheia.
\newcommand{\setfullscreen}{\hypersetup{pdfpagemode=FullScreen}}

% Hide beamer navigation simbols
\beamertemplatenavigationsymbolsempty

%
% Standard frames
%

% coverframe
\newcommand{\coverframe}{%
    \begin{frame} %
        \titlepage %
    \end{frame} %
}

% finalframe{email}
\newcommand{\finalframe}[2][Thank you!]{%
    \begin{frame}%
        \begin{flushright}%
            \huge \textbf{#1}%
            \vfill%
            \large \textbf{#2}%
        \end{flushright}%
    \end{frame}%
}

% bigtitle{title}
\newcommand{\bigtitle}[1]{%
    \begin{frame}%
        \begin{center}%
            \Huge {#1}%
        \end{center}%
    \end{frame}%
}

% citation{cite}{author}
\renewcommand{\citation}[2]{%
    \begin{frame}%
        \begin{center}%
            \vspace{1cm}
            \large \textit{"#1"}\\%
            \vspace{1cm}
            \footnotesize {#2}%
        \end{center}%
    \end{frame}%
}

% bigimage{file}
\newcommand{\bigimage}[2][1.0]{%
    {%
        \usebackgroundtemplate{}%
        \begin{frame}%
            {%
            \makebox[\textwidth][c]{%
              \includegraphics[height=#1\paperheight, width=#1\paperwidth,%
                               keepaspectratio]{#2}%
              }%
            }%
        \end{frame}%
    }%
}


\loadtheme{photoroll}

\title{Iluminação}
\subtitle{Conceitos Básicos de Fotografia Digital}
\author{}
\institute{Rafael\textbf{Jeffman}\\\tiny{F O T O G R A F I A}}
\date{Abril de 2018}

\begin{document}

%01
\coverframe

%02
\begin{frame}
  \frametitle{Qualidade da Luz}
  \begin{itemize}
      \item Luz Dura
      \begin{itemize}
          \item Forma sombras bem marcadas e alto contraste.
          \item Causada por uma fonte de luz pequena e intensa.
      \end{itemize}

      \item Luz Suave
      \begin{itemize}
          \item Forma sombras mal definidas ou inexistentes, baixo contraste.
          \item Causada por uma fonte de luz grande e difusa.
      \end{itemize}
  \end{itemize}
\end{frame}

%03
\imagevertical{Luz Dura}{images/hardlight.jpg}

%04
\imagevertical{Luz Suave}{images/softlight.jpg}

%05
\imagevertical{Luz Suave}{images/natural-soft.jpg}

%06
\begin{frame}
  \frametitle{Direção da Luz}
  \begin{itemize}
      \item Luz de Cima
      \begin{itemize}
          \item Orientação natural da luz.
      \end{itemize}
      \item Luz de Baixo
      \begin{itemize}
          \item Luz utilizada em filmes de terror.
      \end{itemize}
  \end{itemize}
\end{frame}

%07
\imagevertical{Luz de Cima}{images/noon.jpg}

%08
\imagevertical{Luz de Baixo}{images/light_from_below.jpg}

%09
\begin{frame}
    \frametitle{Direção da Luz}
    \begin{itemize}
        \item Luz Frontal
        \begin{itemize}
            \item Gera poucas sombras e deixa o objeto mais plano.
        \end{itemize}
        \item Luz Lateral
        \begin{itemize}
            \item Destaca formas e texturas.
        \end{itemize}
        \item Contraluz
        \begin{itemize}
            \item Cria silhuetas, reduz detalhes.
        \end{itemize}
    \end{itemize}
\end{frame}

%10
\imagevertical{Luz Frontal}{images/front-light.jpg}

%11
\imagevertical{Luz \textit{três-quartos}}{images/three-quarter.jpg}

%12
\imagevertical{Luz Lateral}{images/texture.jpg}

%13
\imagevertical{Contraluz}{images/correct_exposure.jpg}

%14
\begin{frame}
  \frametitle{Luz Natural}
  \begin{itemize}
      \item Vem sempre de cima.
      \item A suavidade depende da quantidade de nuvens.
      \item Varia de cor ao longo do dia.
      \item Mais \textit{quente} onde a luz é direta, mais \textit{fria} onde a
      luz é indireta (sombras).
  \end{itemize}
\end{frame}

%15
\imagevertical{Luz Natural}{images/natural-light.jpg}

%16
\begin{frame}
  \frametitle{Luz Artificial Contínua}
  \begin{itemize}
      \item Emitidas por lâmpadas incandescentes (tungstênio e halógenas) ou fluorescentes.
      \item A temperatura de cor varia de acordo com o tipo de lampada.
      \item Quando utilizam-se as luzes existentes, normalmente, a iluminação vem de cima.
      \item Para efeitos fotográficos, nem a intensidade, nem a cor variam ao longo do tempo.
  \end{itemize}
\end{frame}

%17
\begin{frame}
  \frametitle{Luz do Flash}
  \begin{itemize}
      \item Luz branca, intensa, de curta duração, boa reprodução de cores.
      \item Permite o controle da intensidade e da posição.
      \item Luz dura, pequena, mas que permite a utilização de modificadores.
  \end{itemize}
\end{frame}

%18
\imagevertical{Luz do Flash}{images/flash-light.jpg}

%19
\begin{frame}
  \frametitle{Luzes Chave e Secundária}
  \begin{itemize}
      \item A luz chave (\textit{key light}), ou primária, é a luz que ilumina a cena,
      dando o clima geral.
      \item As luzes secundárias iluminam outras partes específicas da cena, ou são
      utilizadas para reduzir o contraste.
  \end{itemize}
\end{frame}

%20
\imagevertical{Luz Chave e Secundária}{images/three-quarter.jpg}

%21
\begin{frame}
  \frametitle{Luz de Preenchimento}
  \begin{itemize}
      \item Tipo de iluminação utilizada para reduzir as sombras, o contraste.
      \item Pode ser obtida com rebatedores, flashes ou luzes contínuas.
      \item O uso do flash embutido é bom para esse fim.
  \end{itemize}
\end{frame}

%22
\imagevertical{Luz de Preenchimento}{images/fill-light.jpg}

%23
\begin{frame}
  \frametitle{Flash e Luz Disponível}
  \begin{itemize}
      \item Podemos misturar a luz disponível com o flash de várias formas.
      \item A iluminação pode não ficar natural, devido a mais de uma fonte de luz.
      \item A variação de cor pode dificultar a iluminação da cena.
      \item Em geral, utilizamos o flash como luz principal e o ambiente como
      luz secundária.
  \end{itemize}
\end{frame}

%24
\imagevertical{Flash e Luz Disponível}{images/flash-sun.jpg}

%25
\begin{frame}
  \frametitle{Controle da Luz de Flash}
  \begin{itemize}
      \item Até a velocidade de sincronismo, imaginamos o flash como uma luz instantânea.
      \item O controle do \textbf{tempo de exposição} influencia a luz ambiente, mas não
      a luz de flash.
      \item O controle do \textbf{diafragma} influencia a luz ambiente e a quantidade de
      luz do flash que atinge o sensor.
      \item A sensibilidade (\textbf{ISO}) também influencia na exposição do flash.
  \end{itemize}
\end{frame}

%26
\begin{frame}
  \frametitle{Sincronismo de Flash}
  \begin{itemize}
      \item Velocidade máxima na qual o flash pode ser utilizado em uma câmera
      específica.
      \item Alguns flashes possuem um modo de \textit{alta-velocidade}, que diminui
      a potência do flash.
      \item Utilizar o flash em velocidades mais altas (sem o modo de alta-velocidade)
      cria faixas escuras na parte inferior da foto (ou fotos sub-expostas).
  \end{itemize}
\end{frame}

%27
\finalframe[Vamos gastar o obturador!]{rafasgj@gmail.com}

%28
\begin{frame}
    \frametitle{Bibliografia}
    \begin{itemize}
        \item \textbf{The Strobist: Lighting 101}. David Hobby. 2006.
        \hspace{2cm} \url{https://strobist.blogspot.com/2006/03/lighting-101.html}
        \item \textbf{O novo manual de Fotografia}. John Hedgecoe. Senac, 2005.
        \item \textbf{Existing Light Techniques For Wedding and Portrait Photography}.
        Bill Hurter. Amherst Media, 2008.
        \item \textbf{O Olho do Fotógrafo}. Michael Freeman. Bookman, 2012.
    \end{itemize}
\end{frame}

\end{document}
