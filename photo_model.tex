\documentclass[aspectratio=169,14pt,usenames,dvipsnames]{beamer}

\usepackage[utf8]{inputenc}
\usepackage{fontspec}
\usepackage{enumitem}
\usepackage{calc}

\usepackage{datetime}
\newcommand\builddate{%
   \ifcase \month%
        \or Janeiro%
        \or Fevereiro%
        \or Março%
        \or Abril%
        \or Maio%
        \or Junho%
        \or Julho%
        \or Agosto%
        \or Setembro%
        \or Outubro%
        \or Novembro%
        \or Dezembro%
    \fi\space\number\year%
}

\newcommand{\loadtheme}[1]{%
    \input{themes/#1}%
}
\newcommand{\presentationlanguage}[1]{%
    \usepackage[#1]{babel}%
}

\newcommand{\usecodingsamples}[1]{%
    \usepackage{listings}%
    \input{listings/#1}%
}

% Configura a apresentação para ser executada em tela cheia.
\newcommand{\setfullscreen}{\hypersetup{pdfpagemode=FullScreen}}

% Hide beamer navigation simbols
\beamertemplatenavigationsymbolsempty

%
% Standard frames
%

% coverframe
\newcommand{\coverframe}{%
    \begin{frame} %
        \titlepage %
    \end{frame} %
}

% finalframe{email}
\newcommand{\finalframe}[2][Thank you!]{%
    \begin{frame}%
        \begin{flushright}%
            \huge \textbf{#1}%
            \vfill%
            \large \textbf{#2}%
        \end{flushright}%
    \end{frame}%
}

% bigtitle{title}
\newcommand{\bigtitle}[1]{%
    \begin{frame}%
        \begin{center}%
            \Huge {#1}%
        \end{center}%
    \end{frame}%
}

% citation{cite}{author}
\renewcommand{\citation}[2]{%
    \begin{frame}%
        \begin{center}%
            \vspace{1cm}
            \large \textit{"#1"}\\%
            \vspace{1cm}
            \footnotesize {#2}%
        \end{center}%
    \end{frame}%
}

% bigimage{file}
\newcommand{\bigimage}[2][1.0]{%
    {%
        \usebackgroundtemplate{}%
        \begin{frame}%
            {%
            \makebox[\textwidth][c]{%
              \includegraphics[height=#1\paperheight, width=#1\paperwidth,%
                               keepaspectratio]{#2}%
              }%
            }%
        \end{frame}%
    }%
}


\loadtheme{photoroll}

\title{Direção de Modelos}
\subtitle{Conceitos Básicos de Fotografia Digital}
\author{}
\institute{Rafael\textbf{Jeffman}\\\tiny{F O T O G R A F I A}}
\date{Abril de 2018}

\begin{document}

%01
\coverframe

%02
\begin{frame}
  \frametitle{A Importância de Posar Bem}
  \begin{itemize}
      \item Pessoas ficam nervosas na frente da câmera, sentem-se intimidadas.
      \item Modelos iniciantes confiam em fotógrafos para guiá-las em poses bem sucedidas.
      \item Modelos experientes confiam em fotógrafos para ajustar suas poses.
      \item O trabalho do \textbf{fotógrafo} é fazer a pessoa se sentir bem.
  \end{itemize}
\end{frame}

%03
\begin{frame}
  \frametitle{Um \textit{Book} para a Modelo}
  \begin{itemize}
      \item Explore as principais qualidades da modelo.
      \item Mostre aquilo que ela gosta em si mesma.
      \item Esconda atributos que ela não gosta.
      \item O propósito do \textit{book} é mostrar uma grande variedade de poses.
  \end{itemize}
\end{frame}

%04
\begin{frame}
  \frametitle{Prepare-se para o Ensaio}
  \begin{itemize}
      \item Tenha uma ideia clara de qual o seu objetivo para as poses.
      \item Sabendo exatamente o que você quer, fica mais fácil dirigir a modelo.
      \item Tenha exemplos de poses para a modelo.
  \end{itemize}
\end{frame}

%05
\begin{frame}
  \frametitle{Comunique seu Conceito}
  \begin{itemize}
      \item Uma vez definido, comunique o conceito do ensaio à modelo antes de
      iniciar o trabalho.
      \item Seja claro nas instruções, não exija que a modelo tome as decisões de poses.
      \item Não mude muito a pose da modelo para cada foto. De preferência, um elemento por foto.
  \end{itemize}
\end{frame}

%06
\begin{frame}
  \frametitle{Trabalhando com Pessoas}
  \begin{itemize}
      \item Fotografar pessoas não é só registrar uma imagem, é impressionar.
      \item Ofereça bastante apoio positivo, evite comentários negativos.
      \item Olhe para a modelo, não para a câmera.
      \item Exposições digitais são baratas, não economize, não olhe para o LCD.
      \item A modelo irá cansar antes do fotógrafo, ela está tensa, deixe-a descansar e relaxar.
  \end{itemize}
\end{frame}

%07
\begin{frame}
  \frametitle{Desafios Técnicos}
  \begin{itemize}
      \item Além da fotografia, o \textit{book} envolve uma parte importante da edição digital.
      \item Criar um estilo é importante, e para isso, uma equipe de apoio (maquiagem, cabelo, locação) ajuda muito.
      \item Reproduzir o conceito imaginado pode ser um grande desafio, com poses, iluminação e técnica fotográfica.
  \end{itemize}
\end{frame}

%08
\begin{frame}
  \frametitle{Composição}
  \begin{itemize}
      \item A composição no \textit{book} deve guiar os olhos à modelo.
      \item Como ocidentais, "lemos" a imagem da esquerda para a direita.
      \item O olho tende a procurar áreas contrastadas, e seguir linhas.
      \item Cortes justo acentuam a figura humana, cortes abertos relacionam a modelo
      ao ambiente.
  \end{itemize}
\end{frame}

%09
\begin{frame}
  \frametitle{Poses}
  \begin{itemize}
      \item Mantenha a modelo com a postura correta. Não a deixe "cair".
      \item Movimento ajuda a romper a tensão da modelo.
      \item Preste atenção aos detalhes, mãos e pés.
      \item Cuide dos cortes, nunca corte nas articulações.
  \end{itemize}
\end{frame}

%10
\begin{frame}
  \frametitle{Gêneros de \textit{Book}}
  \begin{description}
      \item[Fashion] Estilo de fotografia de moda, com produção destacada.
      \item[Glamour] Retrato elegante do corpo humano.
      \item[Editorial] Mostram um conceito, uma tendência.
      \item[Budoir] Fotos \textit{glamour} com conotação sensual, e tendência ao luxo.
  \end{description}
\end{frame}

%11
\begin{frame}
  \frametitle{Cenário}
  \begin{itemize}
      \item Ajuda a mostrar o conceito do \textit{book}.
      \item Locações externas são mais difíceis de controlar.
      \item Locações internas são, em geral, menos atraentes, mas direcionam o foco
      à modelo.
      \item Utilizando lentes longas (50mm ou mais) e claras ($f/2.8$) auxiliam a
      desfocar e tirar atenção do fundo.
  \end{itemize}
\end{frame}

%12
\begin{frame}
  \frametitle{Iluminação}
  \begin{itemize}
      \item Utilizar luz frontal destaca a forma e esconde detalhes.
      \item Luz lateral destaca volume e detalhes.
      \item \textit{High Key} tende a criar um clima leve, inocente.
      \item \textit{Low Key} tende a criar um clima dramático, sombrio.
  \end{itemize}
\end{frame}

%13
\begin{frame}
  \frametitle{Lembre-se}
  \begin{itemize}
      \item Tenha um plano. Se atenha ao plano. Mude o plano, se necessário.
      \item De reforço positivo, elogie.
      \item Oriente a modelo claramente, mostre o que ela deve fazer.
      \item Nunca toque a modelo.
      \item Deixe a modelo relaxada, descansada, interessada.
  \end{itemize}
\end{frame}

%14
\finalframe[Vamos gastar o obturador!]{rafasgj@gmail.com}

%15
\begin{frame}
    \frametitle{Bibliografia}
    \begin{itemize}
        \item \textbf{BOOK: Direção de Modelos para Fotógrafos}. Billy Pegram. Ed. Photos, 2010.
        \item \textbf{Professional Model Portfolios}. Billy Pegram. Amherst Media, 2004.
        \item \textbf{1.000 Poses para Fotos de Mulheres.}. Eliot Siegel. Europa, 2012.
        \item \textbf{Composição}. David Präkel. Bookman, 2010.
    \end{itemize}
\end{frame}

\end{document}
